\section{Notation}

The list of notations used throughout the work follows.

$\vb r$ - the coordinate in three-dimensional space;

$r \equiv \abs{\vb r}$ - the absolute value of $\vb r$.

$\vb k$ - the wave vector in the reciprocal space.

$k \equiv \abs{\vb k}$ - the absolute value of the wave vector $\vb k$.

$\delta(...)$ - the Dirac's $\delta$-function.

$\delta_{...}$ - the Kronecker's $\delta$-symbol.

$V$ - the volume.

$N$ - the number of particles.

$\rho$ - the particle density.

$\eta$ - the packing fraction.

$n(\vb r)$ - the microscopic particle density.

$\hat{\rho}_{\vb k}$ - the Fourier component of the microscopic particle density.

$\rho_{\vb k},$ $\rho_0$ - the collective variables.

$\Xi$ - the grand partition function.

$z$ - the activity.

$\mu$ - the chemical potential.

$\Xi_0,$ $z_0,$ $\mu_0$ - the grand partition function, activity, and chemical potential of the reference system, respectively.

$Z_N$ - the configuration integral.

$\beta$ - the inverse temperature.

$\Lambda$ - the de Broglie thermal wavelength.

$\rho^{(n)}(\vb r^n)$ - the equilibrium $n$-particle density.

$g^{(n)}(\vb r^n)$ - the $n$-particle distribution function.

$h^{(n)}(\vb r^n)$ - the $n$-particle total correlation function.

$\hat{h}^{(n)}(\vb k^n)$ - the Fourier component of the $n$-particle total correlation function.

$U(r_{ij})$ - the full pairwise interaction potential between two particles $i$ and $j$ at distance $r_{ij}$.

$U_N(\vb r^N)$, $U_N$ - the potential energy of the interparticle interaction.

$\Psi(r_{ij})$ - the repulsive part of the full interaction potential.

$\Psi_N(\vb r^N)$, $\Psi_N$ - the potential energy of the short-range repulsive interaction.

$\Phi(r_{ij})$ - the attractive part of the full interaction potential. 

$\Phi_N(\vb r^N)$, $\Phi_N$ - the potential energy of the long-range attractive interaction.

$\sigma$ - the hard-sphere diameter. 

$J(\rho - \hat{\rho})$ - the Jacobian for tranformation from $\hat{\rho}_{\vb k}$ to $\rho_{\vb k}.$

$\omega_{\vb k},$ $\omega_0$ - the variables conjugate to collective variables $\rho_{\vb k},$ $\rho_0.$

$\mathfrak{M}_n(\vb k^n),$ $\mathfrak{m}_n(\vb k^{n-1})$ - the cumulants (semi-invariants).

$\mathfrak{J}(\rho)$ - the Jacobian $J(\rho - \hat{\rho})$ averaged over the reference system.

$\tilde{\mathfrak{J}}(\omega)$ - the part of Jacobian $\mathfrak{J}(\rho)$ dependent on $\omega_{\vb k}.$

\section*{Abbreviations}
The following abbreviations were used throughout the work.

GPF - Grand Partition Function.

HS - Hard Spheres.

MF - Mean Field.

RS - Reference System.